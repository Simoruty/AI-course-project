% !TeX encoding = UTF-8
% !TeX spellcheck = it_IT
% !TeX root = main.tex

\section{Componenti aggiuntive}
Il sistema oltre a fornire i servizi di base necessari a soddisfare l'obiettivo prefissato definito nella sezione \ref{object}, presenta alcune funzionalità aggiuntive atte a migliorare l'esperienza con il sistema. Tra queste funzionalità aggiuntive, le principali sono:
\begin{itemize}
	\item Modulo di spiegazione
	\item Interfaccia grafica java
	\item Rappresentazione visuale dei risultati tramite graphviz
\end{itemize}
Di seguito verrà fatta una breve descrizione per ogni funzionalità.

\subsection{Modulo di spiegazione}
Il sistema, una volta terminata l'elaborazione e mostrato a video i risultati del tag, da la possibilità all'utilizzatore di capire come è stato possibile etichettare quel gruppo di parole in un determinato tag. Per fare ciò, durante l'elaborazione, il sistema man mano che cerca di etichettare un particolare tag (ad esempio \emph{Persona}), asserisce oltre ai vari IDDoc, ListaPrecedenti, ListaSuccessivi, etc. , anche la spiegazione di come si è raggiunto quel particolare tag.

Ad esempio, in caso di tag di persone, all'interno del predicato \emph{tag\_persone}, sarà presente anche:

\begin{prologcode}
 tag_persona(C, N) :-
   ...
   atomic_list_concat(['[PERSONA] Presenza nel documento di:',C,N],
                      ' ',Spiegazione),
   ...
   assertTag(persona(C, N),IDDoc, ListaPrecedenti, ListaSuccessivi,
            Spiegazione, Dipendenze).
\end{prologcode}

Successivamente, qualora l'utente dovesse richiedere la spiegazione di un particolare tag, il sistema, indipendentemente dall'interfaccia utilizzata (per i diversi tipi di approcci interattivi vedere cap. \ref{Interaction}), non farà altro che richiamare il predicato \emph{spiega\_tutto} il cui compito sarà quello di recuperare tutte le spiegazioni dei vari sottotag che compongono il tag di cui si vuole ottenere la spiegazione.
Ad esempio, se nel documento è presente una frase del tipo : \emph{... Richiesta in via chirografaria di 2000\officialeuro ...} , in questo il sistema creerà un tag \emph{richiesta\_valuta(2000, euro, chirografario)}.

Qualora volessimo sapere il come è stato creato questo tag, il sistema risponderà nel seguente modo:

\begin{verbatim}
[RICHIESTA VALUTA] Presenza nella stessa frase della valuta 2000
                   euro e del termine chirografaria
[TIPO_RICHIESTA] Presenza nel documento del termine chirografaria
[VALUTA] Presenza nel documento del numero 2000 preceduto dal 
         simbolo €
\end{verbatim}
\subsection{Interfaccia grafica java}
Tra le feature aggiuntive offerte dal sistema, vi è la presenza di un'interfaccia grafica creata in Java utile al fine di poter migliorare l'usabilità del sistema permettendone l'utilizzo anche a chi non è molto pratico di interfacce a linee di comando. Nel capitolo \ref{Interaction} verrà descritto in dettaglio come si presenta l'interfaccia grafica e un suo esempio di interazione. 
Per permettere la realizzazione delle interfacce grafiche in Java è stato, inoltre, necessario far comunicare Prolog con Java in modo tale da poter permettere uno scambio di informazioni tra i due sistemi; per fare ciò sono state utilizzate due librerie java nate per questo scopo quali JPL e InterProlog di cui si parlerà approfonditamente rispettivamente nelle sezioni \ref{JPL} e \ref{InterProlog}.
\subsection{Rappresentazione visuale dei risultati tramite graphviz}
\clearpage