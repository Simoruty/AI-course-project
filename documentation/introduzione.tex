% !TeX encoding = UTF-8
% !TeX spellcheck = it_IT
% !TeX root = main.tex

\section{Introduzione}

\subsection{Text Analytics}
\nocite{wiki:textMining}
\nocite{gartner:textAnalytics}
\nocite{expertsystem:textAnalytics}
La Text Analytics è un processo per identificare ed estrarre entità (persone, luoghi, organizzazioni, indirizzi, valute, etc.) da dati non strutturati (documenti, pagine web, email, PDF) attraverso l'utilizzo ibrido di tecniche linguistiche, statistiche e di machine learning. 

Questo processo generalmente viene utilizzato per diversi scopi:
\begin{itemize}
	\item \emph{Summarization}: Si riassume il documento in questione individuandone le parole chiave;
	\item \emph{Sentiment Analysis}: si identificano, estraggono, etichettano e rielaborano le informazioni legate ad uno o più brand con l'obiettivo di determinare sia l'attitudine di chi ha pubblicato un contenuto legato alla marca stessa sia la polarità contestuale del contenuto (positiva, neutra, negativa);
	\item \emph{Esplicativa}: Capire a cosa si vuole arrivare con un particolare documento;
	\item \emph{Investigativa}: Capire la causa di uno specifico problema;
	\item \emph{Classificazione}: Classificare il testo in un particolare argomento.
\end{itemize}

L'utilizzo di un sistema esperto per questa tipologia di processi può risultare molto vantaggiosa in quanto diventa possibile automatizzare l'estrazione delle informazioni dal testo non solo in base alla lingua con cui è stato realizzato il testo, ma anche grazie alla base di conoscenza del dominio.

\subsection{Scopo del progetto}
\label{scopo}
L'obiettivo del progetto sarà quello di creare un sistema esperto che sia in grado di estrarre le informazioni salienti da documenti giuridici inerenti istanze fallimentari.