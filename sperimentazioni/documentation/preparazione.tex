\section{Preparazione dei dati}
Per poter fornire i dataset ai 3 sistemi di apprendimento, è necessaria un'opera di riscrittura dei dati in formati riconosciuti dagli stessi.

Di seguito verranno presentate tutte le procedure utilizzate.


\subsection{Aleph}
\label{preparazione:aleph}
Aleph necessita di 3 file di input:
\begin{itemize}
\item file con estensione \verb+.b+ contenente la knowledge base
\item file con estensione \verb+.f+ contenente gli esempi di training positivi
\item file con estensione \verb+.n+ contenente gli esempi di training negativi
\end{itemize}

In questo modo è già possibile avviare Aleph in maniera interattiva e sottoporgli dei casi di test. Tuttavia, per automatizzare il procedimento, abbiamo creato ulteriori due file (un file \verb+.f+ e uno \verb+.n+) che contengono gli esempi di test, rispettivamente, positivi e negativi che il sistema utilizza in maniera automatica se opportunamente configurato.

Il file \verb+.b+ contiene:
\begin{itemize}
\item parametri che modificano l'esecuzione (e ovviamente anche le prestazioni) del sistema
\item le modalità, cioè la descrizione delle relazioni utilizzate nel dataset
\item le determinazioni, ossia le dipendenze della classe dalle relazioni
\item la knowledge base, cioè tutti i fatti del dataset
\end{itemize}

\subsubsection{I file di input}

\subsubsection*{I parametri}

I settaggi vanno espressi con la sintassi:
\begin{verbatim}
     :- set(chiave, valore).
\end{verbatim}

Nello specifico le chiavi da noi impostate sono:

\paragraph{cache\_clauselength} 5

Imposta il limite superiore della lunghezza delle clausole la cui copertura è salvata in cache per usi futuri. (default 3)

\paragraph{caching} true

 Imposta la cache, ovvero salva le clausole e la copertura per usi futuri. (default false)

\paragraph{check\_useless}   true

Specifica se una chiamata a \verb+redundat/2+ dovrebbe essere fatta controllato la ridondanza di letterali nella clausola. (default false)

\paragraph{clauselength}   8

   Imposta la lunghezza massima delle clausole che costruiscono la teoria a 8.
 
\paragraph{depth}   15

   Imposta a 15 il limite superiore di profondità al quale il theorem-proving procede. (default 10) 

\paragraph{i}   10

Imposta a 10 il numero massimo di variabili che può contenere una clausola. (default 2)

\paragraph{minacc}   0.0

   Imposta un limite minimo di accuratezza per una clausola accettabile. (default 0.0) 

\paragraph{minpos}   2

Serve ad escludere teorie che coprono
meno di due esempi. Fissandolo a 2 il
sistema non fornirà teorie ground, che
infatti sono da evitare. (default 1)


\paragraph{nodes}   50000

Aumenta il livello di profondità nell’albero di decisione per raggiungere la soluzione. (default 5000)

\paragraph{noise}   0

Soglia che migliora la teoria in quanto
evita che essa copra qualunque esempio
negativo. In altri termini si sta dicendo
che la teoria può coprire al massimo zero
esempi negativi.

\paragraph{record}   true

   Il sistema deve registrare il log della sua esecuzione in un file esterno.

\paragraph{recordfile}   './elsevier\_f0.log'

   Imposta il percorso e il nome del file che conterrà il log dell'esecuzione del sistema. Il nome file è nella forma:\\ \emph{dataset}~\verb+_f+~\emph{numeroFold}~\verb+.log+ 

\paragraph{rulefile}   './elsevier\_f0.rul'

         Imposta il percorso e il nome del file che conterrà le regole generate dal sistema. Il nome file è nella forma:\\ \emph{dataset}~\verb+_f+~\emph{numeroFold}~\verb+.rul+ 

\paragraph{test\_pos}   './elsevier\_f0\_test.f'

      Imposta il percorso e il nome del file che contiene gli esempi di test positivi sui quali il sistema dovrà automaticamente provare l'efficacia della teoria appresa. Il nome file è nella forma:\\ \emph{dataset}~\verb+_f+~\emph{numeroFold}~\verb+_test.f+

\paragraph{test\_neg}   './elsevier\_f0\_test.n'

      Imposta il percorso e il nome del file che contiene gli esempi di test negativi sui quali il sistema dovrà automaticamente provare l'efficacia della teoria appresa. Il nome file è nella forma:\\ \emph{dataset}~\verb+_f+~\emph{numeroFold}~\verb+_test.n+
      
\paragraph{thread}   8

   Utilizza 8 thread per beneficiare del calcolo parallelo sui processori moderni. La sperimentazione è stata avviata su un notebook con processore Intel$^{\scriptsize{\copyright{}}}$ i7$^{\scriptsize{\copyright{}}}$, dotato di 4 core e tecnologia HyperThreading$^{\scriptsize{\copyright{}}}$ in grado di supportare fino a 8 thread.
 
\paragraph{verbosity}   0

Imposta la quantità di output prodotta dal sistema. Il valore $0$ indica la stampa solo di output essenziale.

\subsubsection*{I modi}
I modi servono a descrivere le relazioni che verranno utilizzate nella formazione delle clausole.

\begin{verbatim}
     :- modeh(RecallNumber, Mode).
\end{verbatim}
dichiara che \verb+Mode+ può essere utilizzato nella testa delle clausole ipotizzate. \verb+RecallNumber+ è un parametro, numero intero oppure *, che indica il numero massimo di chiamate al predicato.

\begin{verbatim}
     :- modeb(RecallNumber, Mode).
\end{verbatim}
dichiara che \verb+Mode+ può essere utilizzato nel corpo delle clausole ipotizzate. \verb+RecallNumber+ è un parametro, numero intero oppure *, che indica il numero massimo di chiamate al predicato.

\subsubsection*{Le determinazioni}
Ogni riga del tipo:
\begin{verbatim}
     :- determination(T/1, P/2).
\end{verbatim}
dichiara che il predicato P può essere utilizzato per costruire una ipotesi su T.
Il primo argomento è il nome e l'arietà del predicato target, cioè il predicato che apparirà nella testa delle ipotesi. Il secondo argomento è il nome e l'arietà di un predicato che può apparire nel corpo di quella ipotesi.

\subsection{•}


\subsection{FOIL}
FOIL prevede un solo file di input con estensione \verb+.d+.
Il file è strutturato in tre
parti separate da una linea vuota. Queste parti sono composte, nell'ordine, da:
\begin{itemize}
\item tutti i tipi di dati utilizzati
\item la definizione delle relazioni utilizzate
\item esempi di test sui quali valutare l'accuratezza
\end{itemize}

I tipi di dati possono essere definiti in maniera
intensionale quando sono di tipo numerico, altrimenti devono essere definiti in maniera estensionale. 
