\section{Preparazione dei dati}
Per poter fornire i dataset ai 3 sistemi di apprendimento, è necessaria un'opera di riscrittura dei dati in formati riconosciuti dagli stessi.

Di seguito verranno presentate tutte le procedure utilizzate.


\subsection{Aleph}
Aleph necessita di 3 file di input:
\begin{itemize}
\item file con estensione \verb+.b+ contenente la knowledge base
\item file con estensione \verb+.f+ contenente gli esempi di training positivi
\item file con estensione \verb+.n+ contenente gli esempi di training negativi
\end{itemize}

In questo modo è già possibile avviare Aleph in maniera interattiva e sottoporgli dei casi di test. Tuttavia, per automatizzare il procedimento, abbiamo creato ulteriori due file (un file \verb+.f+ e uno \verb+.n+) che contengono i casi di testing, rispettivamente, positivi e negativi che il sistema utilizza in maniera automatica se opportunamente configurato.

\subsubsection{Sintassi file .b}
Il file \verb+.b+ contiene:
\begin{itemize}
\item settaggi che modificano l'esecuzione (e ovviamente anche le prestazioni) del sistema
\item la descrizione delle relazioni utilizzate nel dataset
\item le dipendenze della classe dalle relazioni
\item tutti i fatti del dataset
\end{itemize}

I settaggi vanno espressi con la sintassi:
\begin{verbatim}
     :- set(chiave, valore).
\end{verbatim}

Nello specifico le chiavi da noi impostate sono:

\paragraph{cache\_clauselength} 5

Numero di pagine del Documento 

\paragraph{caching} true

 • 

\paragraph{check\_useless}   true

   • 

\paragraph{clauselength}   8

   Imposta la lunghezza massima delle clausole che costruiscono la teoria a 8.
 

\paragraph{depth}   15

   • 

\paragraph{i}   10

   • 

\paragraph{minacc}   0.0

   • 

\paragraph{minpos}   2

   • 

\paragraph{nodes}   50000

   • 

\paragraph{noise}   0

   • 

\paragraph{record}   true

   • 

\paragraph{recordfile}   './elsevier\_f0.log'

   • 

\paragraph{rulefile}   './elsevier\_f0.rul'

   • 

\paragraph{test\_pos}   './elsevier\_f0\_test.f'

   • 

\paragraph{test\_neg}   './elsevier\_f0\_test.n'

   • 

\paragraph{thread}   8

   Utilizza 8 thread per beneficiare del calcolo parallelo sui processori moderni.
   
 


\paragraph{verbosity}   0

Imposta la quantità di output prodotta dal sistema 
