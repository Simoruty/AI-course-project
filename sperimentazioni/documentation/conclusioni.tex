\section{Conclusioni e sviluppi futuri}

In questo documento sono stati confrontati tre sistemi di apprendimento su un task di classificazione binaria di layout di documenti testuali.

I dataset presi in considerazione sono stati quattro, ai quali è stato aggiunto un quinto generato dalla rappresentazione discretizzata su alcuni valori numerici di uno dei precedenti quattro. Ogni dataset è stato suddiviso in 10 fold, secondo uno schema di campionamento casuale stratificato.

I risultati, corroborati da test statistici, mostrano una sostanziale similarità tra ALEPH e Progol su tutti i dataset. 
FOIL invece, pur evidenziando le migliori prestazioni in termini di tempo di esecuzione, mostra un'accuratezza minore.

Per il futuro, si potrebbero condurre delle sperimentazioni per analizzare l'influenza dei parametri da noi bloccati sull'accuratezza della classificazione.