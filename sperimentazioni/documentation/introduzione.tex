% !TEX root = main.tex
\section{Introduzione}

In questo documento analizzeremo tre sistemi di apprendimento dedicati al task di classificazione binaria, col fine di valutarli e confrontarli.

La sperimentazione verrà condotta su quattro differenti dataset e avrà l'obiettivo di classificare il layout di documenti testuali scientifici.

I sistemi di apprendimento da analizzare sono stati sviluppati con la tecnica dell’ILP (Inductive Logic Programming).

La ILP è un tipo di programmazione logica che sfrutta la potenza espressiva della logica del prim'ordine (Tramite le clausole di Horn) per la rappresentazione delle osservazioni, delle ipotesi e della teoria di fondo ed inoltre eredita gli obiettivi dell’apprendimento induttivo.