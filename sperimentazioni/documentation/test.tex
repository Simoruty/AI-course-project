\section{Test di Valutazione}
Per poter analizzare i dati da un punto di vista statistico in modo da capire quale dei tre sistemi risulti essere migliore in un task di classificazione, è necessario utilizzare un test delle ipotesi che sia in grado di rigettare o confermare l'ipotesi nulla.

L'ipotesi nulla da rigettare in particolare sarà:
$$H_0 : \mu_{ALEPH} = \mu_{PROGOL} = \mu_{FOIL} $$
Simmetricamente l'ipotesi alternativa sarà:
$$H_1 : \mu_{ALEPH} \neq \mu_{PROGOL} \neq \mu_{FOIL} $$

Tra le diverse metriche calcolate per i diversi algoritmi, l'accuratezza è la metrica utilizzata affinché venga verificato il test delle ipotesi.

Poiché per ogni dataset è stata ripetuta la sperimentazione, il test delle ipotesi sarà condotto per ognuno dei cinque dataset separatamente.

Considerato il numero di sistemi da confrontare, il test più adatto per verificare questo tipo di ipotesi è il test parametrico ANOVA (\textbf{AN}alysis \textbf{O}f \textbf{VA}riance). Questo test presuppone che i gruppi di dati provengano da una popolazione normale in ipotesi di omoschedasticità (ossia, la varianza dei gruppi è la stessa di quella della popolazione),condizione veritiera nei dataset a disposizione. Inoltre si è deciso di usare un livello $\alpha$ di soglia pari al 5\%.

Attraverso l'ausilio di un foglio di calcolo costruito ad-hoc, sono stati estratti i risultati del test, di seguito un sunto di ciò che si è ottenuto.

\begin{verbatim}
Elsevier

               ALEPH   PROGOL   FOIL
      $\mu$    0,99     0,99    0,97       Livello \alpha    0,05
   $\sigma$    0,02     0,02    0,03       Quantile Teorico  3,35
     
  Consuntivo   3,66
                               H0 rigettato
\end{verbatim}

\begin{verbatim}
JMLR

               ALEPH   PROGOL   FOIL
      $\mu$    0,98     0,99    0,97       Livello \alpha    0,05
   $\sigma$    0,02     0,02    0,03       Quantile Teorico  3,35
     
  Consuntivo   0,74
                               H0 accettato
\end{verbatim}

\begin{verbatim}
MLJ - Non discretizzato

               ALEPH   PROGOL   FOIL
      $\mu$    0,97     0,96    0,71       Livello \alpha    0,05
   $\sigma$    0,02     0,02    0,06       Quantile Teorico  3,35
     
  Consuntivo   138,92
                               H0 rigettato
\end{verbatim}

\begin{verbatim}
MLJ - Discretizzato

               ALEPH   PROGOL   FOIL
      $\mu$    0,9     0,91    0,95       Livello \alpha    0,05
   $\sigma$    0,05     0,06    0,03       Quantile Teorico  3,35
     
  Consuntivo   3,47
                               H0 rigettato
\end{verbatim}

\begin{verbatim}
SVLN
               ALEPH   PROGOL   FOIL
      $\mu$    0,97     0,97    0,95       Livello \alpha    0,05
   $\sigma$    0,02     0,03    0,03       Quantile Teorico  3,35
     
  Consuntivo   1,70
                               H0 accettato
\end{verbatim}

Analizzando i risultati cosi ottenuti, si evince che solo in due dei quattro dataset (JMLR e SVLN), i tre algoritmi risultano essere staticamente uguali in termini di accuratezza predittiva.