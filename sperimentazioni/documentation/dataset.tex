\section{Dataset utilizzati}


I tre sistemi precedentemente descritti verranno testati su quattro dataset.

\subsection{Dominio}
I dataset sono stati generati per task inerenti alla Document Image Understanding.
Contengono la descrizione, sotto forma di clausole di Horn, di componenti di documenti testuali opportunamente analizzate.

\subsection{Provenienza}
I dataset contengono le informazioni su paper scientifici) provenienti da 4 journal:
\begin{itemize}
\item Elsevier journals (Elsevier)
\item Springer-Verlag Lecture Notes (SVLN)
\item Journal of Machine Learning Research (JMLR)
\item Machine Learning Journal (MLJ).
\end{itemize}


\subsection{Descrizione}
I \emph{documenti} sono tutti composti da una sola \emph{pagina}. Ogni \emph{pagina} contiene un numero variabile di \emph{frame}, ossia rettangoli omogenei all'interno del paper (e.g. titolo, abstract, paragrafo, logo).
Tutte le informazioni contenute nei dataset riguardano una di queste tre entità (documento,pagina,frame) e le relazioni esistenti fra esse.

I predicati utilizzati sono i seguenti:
\begin{table}[htbp]
\centering
\label{tab:predicati}
\small\begin{tabular}{lccl}
\toprule
\addlinespace
Nome proprietà & Arg 1 & Arg 2 & Descrizione \\
\addlinespace
\midrule
\addlinespace
numero\_pagine & Documento & Integer & Numero di pagine del Documento \\ 
pagina\_1 & Documento & Pagina & Prima pagina del documento \\ 
ultima\_pagina & Pagina &  & Ultima pagina del documento\\
frame & Pagina & Frame & Frame all'interno della pagina\\
allineato\_al\_centro\_orizzontale & Frame & Frame & I frame sono allineati orizzontalmente\\
allineato\_al\_centro\_verticale & Frame & Frame & I frame sono allineati verticalmente\\
altezza\_pagina & Pagina & Float & Altezza della pagina\\
larghezza\_pagina & Pagina & Float & Larghezza della pagina\\
altezza\_rettangolo & Frame & Float & Altezza delf rame\\
larghezza\_rettangolo & Frame & Float & Larghezza del Frame\\
ascissa\_rettangolo & Frame & Float & Posizione orizzontale del frame\\
ordinata\_rettangolo & Frame & Float & Posizione verticale del frame\\
on\_top & Frame & Frame & Primo frame sopra al secondo\\
to\_right & Frame & Frame & Primo Frame alla destra del secondo\\
tipo\_immagine & Frame &  & Il frame contiene un'immagine\\
tipo\_linea\_obbliqua & Frame &  & Il frame contiene una linea obliqua\\
tipo\_linea\_orizzontale & Frame &  & Il frame contiene una linea orizzontale\\
tipo\_misto & Frame &  & Il frame contiene componenti varie\\
tipo\_testo & Frame &  & Il frame contiene solo testo\\
tipo\_vuoto & Frame &  & Il frame è vuoto\\
\addlinespace
\bottomrule 
\end{tabular}
\end{table}

Solo nel caso del dataset \emph{MLJ}, abbiamo avuto a disposizione un file contenente la discretizzazione delle 4 componenti numeriche delle seguenti proprietà: 
\begin{itemize}
\item altezza\_rettangolo
\item larghezza\_rettangolo
\item ascissa\_rettangolo
\item ordinata\_rettangolo
\end{itemize}.

In questo caso abbiamo eliminato le suddette proprietà e aggiunto nuove, secondo la seguente tabella:

\begin{table}[htbp]
\centering
\label{tab:predicati}
\small\begin{tabular}{lccl}
\toprule
\addlinespace
Nome proprietà & Arg 1 & Arg 2 & Descrizione \\
\addlinespace
\midrule
\addlinespace
height\_smallest & Frame & • \\ 
height\_very\_very\_small & Frame & • \\ 
height\_very\_small & Frame & • \\ 
\addlinespace
\bottomrule 
\end{tabular}
\end{table}


 

 dei documenti testuali e le loro componenti.
I quattro dataset rappresentano una suddivisione in quattro classificazioni binarie di una classificazione multiclasse.

\begin{table}[htbp]
	\centering
	\begin{tabular}{cccc}
		\multirow{2}{*}{Elsevier} & + 61 & \multirow{2}{*}{Tot: 353}& \multirow{2}{*}{\%+ 17\%} \\
		 & -292 &  & \\
		 \multirow{2}{*}{JMLR} & +100 & \multirow{2}{*}{Tot: 353} & \multirow{2}{*}{\%+ 28\%} \\
		 & -253 & & \\
 		 \multirow{2}{*}{MLJ} & +93 & \multirow{2}{*}{Tot: 273} & \multirow{2}{*}{\%+ 35\%} \\
 		 & -180 & & \\
 		 \multirow{2}{*}{SVLN} & +70 & \multirow{2}{*}{Tot: 353} & \multirow{2}{*}{\%+ 20\%} \\
 		 & -283 & & \\
	\end{tabular}%
	\label{tab:}
\end{table}

\begin{table}[htbp]
	\label{tab:datasets}
	\centering
	%{l r r r}
\begin{tabular}{c@{\qquad}r@{\qquad}r@{\qquad}c}
\toprule
\addlinespace
Dataset & \#Positive $(\%)$ & \#Negative $(\%)$ & \#Examples \\
\addlinespace
\midrule
\addlinespace
Elsevier & $61$~~($17\%$) & $292~~(83\%)$ & $353$ \\
JMLR     & $100~~(28\%)$ & $253~~(72\%)$ & $353$ \\
MLJ      & $122~~(35\%)$ & $231~~(65\%)$ & $353$ \\
SVLN     & $70~~(20\%)$ & $283~~(80\%)$ & $353$ \\
\addlinespace
\bottomrule
\end{tabular}
\end{table}



\subsection{ELSEVIER}
\begin{table}[htbp]
	\centering
		\begin{tabular}{l@{\qquad}*{10}{r}}
		\toprule
\addlinespace
			Fold &  0 &  1 &  2 &  3 &  4 &  5 &  6 &  7 &  8 &  9 \\
\addlinespace
\midrule
\addlinespace
\#Positive  & 6  & 6  &  6 &  6 &  6 &  6 &  6 &  6 &  6 &  7 \\
\#Negative  & 29 & 29 & 29 & 29 & 29 & 29 & 29 & 30 & 30 & 29 \\
\#Examples & 35 & 35 & 35 & 35 & 35 & 35 & 35 & 36 & 36 & 36 \\
\addlinespace
\bottomrule
		\end{tabular}
	\label{tab:Elsevier}
\end{table}


\subsection{JMLR}
\begin{table}[htbp]
	\centering
		\begin{tabular}{c | ccccccccccc}
			FOLD &  0 &  1 &  2 &  3 &  4 &  5 &  6 &  7 &  8 &  9 \\ \hline
			Doc  & 35 & 35 & 35 & 35 & 35 & 35 & 35 & 36 & 36 & 36 \\
			Pos  & 10  & 10  &  10 &  10 &  10 &  10 &  10 &  10 &  10 &  10 \\
			Neg  & 25 & 25 & 25 & 25 & 25 & 25 & 25 & 26 & 26 & 26 \\
		\end{tabular}%
	\label{tab:JMLR}
\end{table}
\subsection{MLJ}
\begin{table}[htbp]
	\centering
		\begin{tabular}{c | cccccccccc}
			FOLD &  0 &  1 &  2 &  3 &  4 &  5 &  6 &  7 &  8 &  9 \\ \hline
			Doc  & 35 & 35 & 35 & 35 & 35 & 35 & 35 & 36 & 36 & 36 \\
			Pos  & 12  & 12  &  12 &  12 &  12 &  12 &  12 &  12 & 13 & 13 \\
			Neg  & 23 & 23 & 23 & 23 & 23 & 23 & 23 & 24 & 23 & 23 \\
		\end{tabular}
	\label{tab:MLJ}
\end{table}


\subsection{SVLN}
\begin{table}[htbp]
	\centering
		\begin{tabular}{c|cccccccccc}
			FOLD & 0 &  1 &  2 &  3 &  4 &  5 &  6 &  7 &  8 &  9 \\\hline
			Doc  & 35 & 35 & 35 & 35 & 35 & 35 & 35 & 36 & 36 & 36 \\
			Pos  & 7  & 7  &  7 &  7 &  7 &  7 &  7 &  7 &  7 &  7 \\
			Neg  & 28 & 28 & 28 & 28 & 28 & 28 & 28 & 29 & 29 & 29 \\
		\end{tabular}%
	\label{tab:SVLN}
\end{table}
