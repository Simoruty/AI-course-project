\section{Dataset utilizzati}

I tre sistemi precedentemente descritti verranno testati su quattro dataset.

\subsection{Dominio e provenienza}
I dataset sono stati generati per task inerenti alla Document Image Understanding.
Contengono la descrizione, sotto forma di clausole di Horn, di componenti di documenti testuali opportunamente analizzate.

I dataset contengono le informazioni su paper scientifici provenienti da 4 journal:
\begin{itemize}
\item Elsevier journals (Elsevier)
\item Springer-Verlag Lecture Notes (SVLN)
\item Journal of Machine Learning Research (JMLR)
\item Machine Learning Journal (MLJ).
\end{itemize}

\subsection{Descrizione}
I \emph{documenti} sono tutti composti da una sola \emph{pagina}. Ogni \emph{pagina} contiene un numero variabile di \emph{frame}, ossia rettangoli all'interno del paper (e.g. titolo, abstract, paragrafo, logo).

Tutte le informazioni contenute nei dataset riguardano una di queste tre entità (documento, pagina, frame) e le relazioni esistenti fra esse.

I predicati utilizzati sono riportati in tabella \ref{tab:predicati}
\begin{table}[htbp]
\centering
\label{tab:predicati}
\footnotesize\begin{tabular}{lccl}
\toprule
\addlinespace
Nome predicati & Arg 1 & Arg 2 & Descrizione \\
\addlinespace
\midrule
\addlinespace
numero\_pagine & Doc & Int & Numero di pagine del Documento \\ 
pagina\_1 & Doc & Page & Prima pagina del documento \\ 
ultima\_pagina & Page & - & Ultima pagina del documento\\
frame & Page & Frame & Frame all'interno della pagina\\
allineato\_al\_centro\_orizzontale & Frame & Frame & I frame sono allineati orizz.\\
allineato\_al\_centro\_verticale & Frame & Frame & I frame sono allineati vert.\\
altezza\_pagina & Page & Float & Altezza della pagina\\
larghezza\_pagina & Page & Float & Larghezza della pagina\\
altezza\_rettangolo & Frame & Float & Altezza delf rame\\
larghezza\_rettangolo & Frame & Float & Larghezza del Frame\\
ascissa\_rettangolo & Frame & Float & Posizione orizzontale del frame\\
ordinata\_rettangolo & Frame & Float & Posizione verticale del frame\\
on\_top & Frame & Frame & Primo frame sopra al secondo\\
to\_right & Frame & Frame & Primo Frame alla destra del secondo\\
tipo\_immagine & Frame & - & Il frame contiene un'immagine\\
tipo\_linea\_obbliqua & Frame & - & Il frame contiene una linea obliqua\\
tipo\_linea\_orizzontale & Frame & - & Il frame contiene linea orizzontale\\
tipo\_misto & Frame & - & Il frame contiene componenti varie\\
tipo\_testo & Frame & - & Il frame contiene solo testo\\
tipo\_vuoto & Frame & - & Il frame è vuoto\\
\addlinespace
\bottomrule 
\end{tabular}
\end{table}

Solo nel caso del dataset \emph{MLJ}, abbiamo avuto a disposizione un file contenente gli intervalli di discretizzazione delle 4 componenti numeriche delle seguenti proprietà
\begin{itemize}
\item altezza\_rettangolo
\item larghezza\_rettangolo
\item ascissa\_rettangolo
\item ordinata\_rettangolo
\end{itemize}
.

Solo per \emph{MLJ}, dunque, abbiamo potuto effettuare due sperimentazioni (con valori discretizzati e non). Per farlo, abbiamo eliminato le 4 proprietà numeriche e aggiunto 24 nuove proprietà.

\verb+altezza_rettangolo+ è stata divisa in $11$ bin, \verb+larghezza_rettangolo+ in $7$ bin, \verb+ascissa_rettangolo+ e \verb+ordinata_rettangolo+ in $3$ bin.

I nomi delle proprietà aggiunte sono riportate in tabella \ref{tab:discretizzazione}.

\begin{table}[htbp]
\centering
\label{tab:discretizzazione}
\small\begin{tabular}{lc}
\toprule
\addlinespace
Nome proprietà & Argomento \\
\addlinespace
\midrule
\addlinespace
height\_smallest & Frame \\ 
height\_very\_very\_small & Frame \\ 
height\_very\_small & Frame \\ 
height\_small & Frame \\ 
height\_medium\_small & Frame \\ 
height\_medium & Frame \\ 
height\_medium\_large & Frame \\ 
height\_large & Frame \\ 
height\_very\_large & Frame \\ 
height\_very\_very\_large & Frame \\ 
height\_largest & Frame \\ 
\midrule
width\_very\_small & Frame \\ 
width\_small & Frame \\ 
width\_medium\_small & Frame \\ 
width\_medium & Frame \\ 
width\_medium\_large & Frame \\ 
width\_large & Frame \\ 
width\_very\_large & Frame \\ 
\midrule
pos\_left & Frame \\ 
pos\_center & Frame \\ 
pos\_right & Frame \\
\midrule
pos\_upper & Frame \\ 
pos\_middle & Frame \\ 
pos\_lower & Frame \\  
\addlinespace
\bottomrule 
\end{tabular}
\end{table}


 
Ci sono gli stessi $353$ documenti in tutti i dataset. Ognuno di questi risulta essere esempio positivo in solo uno dei quattro dataset e, in maniera esclusiva, negativo per gli altri tre.

I quattro dataset rappresentano una suddivisione in quattro classificazioni binarie di una classificazione multiclasse.


\begin{table}[htbp]
	\label{tab:datasets}
	\centering
	%{l r r r}
\begin{tabular}{c@{\qquad}r@{\qquad}r@{\qquad}c}
\toprule
\addlinespace
Dataset & \#Positive $(\%)$ & \#Negative $(\%)$ & \#Examples \\
\addlinespace
\midrule
\addlinespace
Elsevier & $61~~(17\%$) & $292~~(83\%)$ & $353$ \\
JMLR     & $100~~(28\%)$ & $253~~(72\%)$ & $353$ \\
MLJ      & $122~~(35\%)$ & $231~~(65\%)$ & $353$ \\
SVLN     & $70~~(20\%)$ & $283~~(80\%)$ & $353$ \\
\addlinespace
\bottomrule
\end{tabular}
\end{table}

Di seguito si riportano le frequenze dei predicati in ciascun dataset. Come spiegato in precedenza, è inutile ripetere i valori per i differenti dataset in quanto ridondanti.

\begin{table}[h!tbp]
\centering
\label{tab:frequenzaPredicati}
\small\begin{tabular}{lr}
\toprule
\addlinespace
Predicato & Frequenza \\
\addlinespace
\midrule
\addlinespace
class\_elsevier * & 61 \\
class\_jmlr * & 100 \\
class\_mlj * & 122 \\
class\_svln * & 70 \\
numero\_pagine & 353 \\
pagina\_1 & 353 \\
ultima\_pagina & 353 \\
frame & 3879 \\
allineato\_al\_centro\_orizzontale & 165 \\
allineato\_al\_centro\_verticale & 5541 \\
altezza\_pagina & 353 \\
larghezza\_pagina & 353 \\
altezza\_rettangolo & 3879 \\
larghezza\_rettangolo & 3879 \\
ascissa\_rettangolo & 3879 \\
ordinata\_rettangolo & 3879 \\
on\_top & 20673 \\
to\_right & 16502 \\
tipo\_immagine & 2 \\
tipo\_linea\_obbliqua & 5 \\
tipo\_linea\_orizzontale & 244 \\
tipo\_misto & 241 \\
tipo\_testo & 3386 \\
tipo\_vuoto & 1 \\
\addlinespace
\bottomrule
\addlinespace
(*) Solo nel proprio dataset & \\
\end{tabular}
\end{table}



\begin{table}[h!tbp]
\centering
\label{tab:frequenzaPredicat}
\small\begin{tabular}{lr}
\toprule
\addlinespace
Predicato & Frequenza \\
\addlinespace
\midrule
\addlinespace
height\_smallest & 269 \\
height\_very\_very\_small & 1515 \\
height\_very\_small & 728 \\
height\_small & 705 \\
height\_medium\_small & 356 \\
height\_medium & 757 \\
height\_medium\_large & 206 \\
height\_large & 173 \\
height\_very\_large & 5 \\
height\_very\_very\_large & 0 \\
height\_largest & 0 \\
width\_very\_small & 45 \\
width\_small & 74 \\
width\_medium\_small & 333 \\
width\_medium & 1584 \\
width\_medium\_large & 784 \\
width\_large & 1817 \\
width\_very\_large & 361 \\
pos\_left & 915 \\
pos\_center & 2575 \\
pos\_right & 357 \\
pos\_upper & 1542 \\
pos\_middle & 1419 \\
pos\_lower & 918 \\
\addlinespace
\bottomrule
\end{tabular}
\caption{Predicati presenti solo nel dataset mlj, quando effettuata la discretizzazione.}
\end{table}