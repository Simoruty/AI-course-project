\section{Sperimentazione}

\subsection{I file eseguibili}
Per ogni fold di ogni dataset, abbiamo creato in maniera automatica uno script \verb+.yap+ che, se eseguito, si occupa di avviare ALEPH, leggere i file di input, generare le regole e scriverle su file.

Il sorgente dello script (esempio tratto dal fold 0 del dataset elsevier) è il seguente:

\begin{prologcode}
#!/usr/local/bin/yap -L --
#
# .
:- consult('../aleph.pl').
:- read_all('elsevier_f0').
:- induce.
:- write_rules('elsevier_f0.rul').
\end{prologcode}

Tutti i $40$ ($4$ dataset $\times$ $10$ fold) script \verb+.yap+ sono avviati in successione da un unico file scritto in linguaggio \emph{Python}.

Esso si occupa anche di tenere traccia dei tempi di esecuzione per un confronto sulle performance dei sistemi confrontati e di scrivere l'output delle esecuzioni in file \verb+.out+.

Il codice dello script è qui riportato:

\begin{pythoncode}
#!/usr/bin/python

import sys
import os
import subprocess
from time import gmtime, strftime, localtime
from datetime import datetime

cmd = "chmod +x **/*.yap"
subprocess.call(cmd, shell=True)
if not os.path.exists("./result/"):
    os.makedirs("./result/")
for dataset in ["elsevier", "jmlr", "mlj", "svln"]:
    print dataset +" started at "+ strftime("%H:%M:%S", localtime())
    sys.stdout.flush()
    startDataset = datetime.now()
    for fold in range(10):
        startTime = datetime.now()
        print "Fold " + str(fold) +" started at "
        print strftime("%H:%M:%S", localtime())
        sys.stdout.flush()
        cmd = "./"+dataset+"/"+dataset+"_f"+str(fold)
        cmd += ".yap -s50000 -h200000 2>&1 > /dev/null"
        subprocess.call(cmd, shell=True)
        fin = open("./result/"+dataset+".summary","a")
        fin.write("\nFold "+str(fold)+"\n")
        fin.close()
        cmd0 = "cat ./"+dataset+"/"+dataset+"_f"+str(fold)
        cmd0 += ".rul >> ./result/"+dataset+".summary"
        subprocess.call(cmd0, shell=True)
        cmd1 = "cat ./"+dataset+"/"+dataset+"_f"+str(fold)
        cmd1 += ".log | grep -B 10 -m 1 '\[Test set summary\]'"
        cmd1 += " >> ./result/"+dataset+".summary"
        subprocess.call(cmd1, shell=True)
        print "Fold " + str(fold) +" ended in: "
        print str(datetime.now()-startTime)
        sys.stdout.flush()
    print dataset+" ended in "+str(datetime.now()-startDataset)
    sys.stdout.flush()
\end{pythoncode}


\subsection{Hardware utilizzato}
\label{hw}
L'hardware utilizzato ai fini della sperimentazione è stato un notebook Acer Aspire v3-571g aventi le seguenti caratteristiche:
\begin{itemize}
	\item \textbf{Processore}: Intel\textregistered Core\texttrademark  i7-3610QM processor (2.3GHz/3.3GHz w/ Turbo Boost)
	\item \textbf{RAM}: 8GB DDR3 1333 MHz
	\item \textbf{Memoria di archiviazione}: 120GB SSD
	\item \textbf{OS}: elementaryOS 0.2 (luna) basato su Ubuntu 12.04.5 LTS (Precise Pangolin)
\end{itemize}

\subsection{Output Sperimentazione}
Di seguito verrà mostrato un estratto di ciò che è stato prodotto dall'esecuzione dei tre algoritmi in particolar modo sul dataset Elsevier sui fold 0 e 1.
\subsubsection{FOIL}
\begin{verbatim}
--- Fold 0

class_elsevier(A) :- 
   pagina_1(A,B), 
   frame(B,C), 
   allineato_al_centro_verticale(C,D),
   allineato_al_centro_verticale(E,D), 
   allineato_al_centro_verticale(F,C), 
   allineato_al_centro_verticale(E,F), 
   allineato_al_centro_verticale(D,H), 
   tipo_linea_orizzontale(C).
class_elsevier(A) :- 
   pagina_1(A,B), 
   frame(B,C), 
   allineato_al_centro_verticale(D,C), 
   tipo_linea_orizzontale(D), 
   allineato_al_centro_verticale(C,E), 
   allineato_al_centro_verticale(E,F), 
   on_top(C,D).
class_elsevier(A) :- 
   pagina_1(A,B), 
   frame(B,C), 
   tipo_misto(C), 
   on_top(C,D), 
   allineato_al_centro_verticale(C,E), 
   on_top(E,F), 
   on_top(D,E), 
   on_top(E,G), 
   allineato_al_centro_verticale(G,F).

Time 3.3 secs

Test relation class_elsevier
(+)	d06090909534026371
Summary: 1 error in 35 trials

--- Fold 1

class_elsevier(A) :- 
   pagina_1(A,B), 
   frame(B,C), 
   allineato_al_centro_verticale(D,C),
   allineato_al_centro_verticale(C,E),
   allineato_al_centro_verticale(E,F),
   allineato_al_centro_verticale(G,D),
   tipo_linea_orizzontale(E).
class_elsevier(A) :- 
   pagina_1(A,B),
   frame(B,C),
   allineato_al_centro_verticale(D,C),
   tipo_linea_orizzontale(D),
   allineato_al_centro_verticale(C,E),
   allineato_al_centro_verticale(E,F),
   on_top(C,D).
class_elsevier(A) :- 
   pagina_1(A,B),
   frame(B,C),
   tipo_misto(C),
   on_top(C,D),
   allineato_al_centro_verticale(C,E),
   on_top(C,E),
   on_top(E,F),
   on_top(D,G),
   tipo_testo(D),
   allineato_al_centro_verticale(G,E),
   tipo_misto(F).

Time 2.9 secs

Test relation class_elsevier
(+)	sciserv37
(+)	sciserv21
Summary: 2 errors in 35 trials
\end{verbatim}
\dots
\subsubsection{PROGOL}
\begin{verbatim}
 --- Fold 0

[theory]

[Rule 1] [Pos cover = 3 Neg cover = 0]
class_mlj(A) :-
   pagina_1(A,B),
   frame(B,C),
   height_very_large(C),
   width_large(C).
[pos-neg] [3]

[Rule 2] [Pos cover = 27 Neg cover = 0]
class_mlj(A) :-
   pagina_1(A,B),
   altezza_pagina(B,842.0),
   frame(B,C),
   tipo_testo(C),
   pos_right(C).
[pos-neg] [27]

[Rule 3] [Pos cover = 13 Neg cover = 0]
class_mlj(A) :-
   pagina_1(A,B),
   frame(B,C),
   tipo_testo(C),
   width_very_small(C),
   pos_upper(C).
[pos-neg] [13]

[Rule 4] [Pos cover = 21 Neg cover = 0]
class_mlj(A) :-
   pagina_1(A,B),
   frame(B,C),
   allineato_al_centro_verticale(C,D),
   width_medium_small(D),
   pos_left(D),
   pos_middle(D).
[pos-neg] [21]

[Rule 5] [Pos cover = 1 Neg cover = 0]
class_mlj(d06060518304000869).
[pos-neg] [1]

[Rule 6] [Pos cover = 19 Neg cover = 0]
class_mlj(A) :-
   pagina_1(A,B),
   frame(B,C),
   on_top(C,D),
   height_smallest(D),
   pos_upper(D).
[pos-neg] [19]

[Rule 7] [Pos cover = 8 Neg cover = 0]
class_mlj(A) :-
   pagina_1(A,B),
   frame(B,C),
   tipo_misto(C),
   height_very_small(C),
   pos_upper(C).
[pos-neg] [8]

[Rule 8] [Pos cover = 1 Neg cover = 0]
class_mlj(d06060510445614393).
[pos-neg] [1]

[Rule 9] [Pos cover = 1 Neg cover = 0]
class_mlj(d06053013213312739).
[pos-neg] [1]

[Rule 10] [Pos cover = 13 Neg cover = 0]
class_mlj(A) :-
   pagina_1(A,B),
   altezza_pagina(B,842.0),
   frame(B,C),
   height_medium_large(C),
   pos_lower(C),
   larghezza_pagina(B,595.0).
[pos-neg] [13]

[Rule 11] [Pos cover = 18 Neg cover = 0]
class_mlj(A) :-
   pagina_1(A,B),
   altezza_pagina(B,842.0),
   frame(B,C),
   height_large(C),
   pos_lower(C).
[pos-neg] [18]

[Rule 12] [Pos cover = 46 Neg cover = 0]
class_mlj(A) :-
   pagina_1(A,B),
   frame(B,C),
   on_top(C,D),
   height_large(D),
   width_large(D),
   width_medium_small(C).
[pos-neg] [46]

[Rule 13] [Pos cover = 3 Neg cover = 0]
class_mlj(A) :-
   pagina_1(A,B),
   frame(B,C),
   allineato_al_centro_orizzontale(C,D),
   pos_center(D),
   pos_middle(D).
[pos-neg] [3]

[Rule 14] [Pos cover = 1 Neg cover = 0]
class_mlj(d06060518380901422).
[pos-neg] [1]

[Rule 15] [Pos cover = 1 Neg cover = 0]
class_mlj(d06060518525802442).
[pos-neg] [1]

[Rule 16] [Pos cover = 1 Neg cover = 0]
class_mlj(d06060518272200636).
[pos-neg] [1]

[Rule 17] [Pos cover = 1 Neg cover = 0]
class_mlj(d06053013220812764).
[pos-neg] [1]

[Rule 18] [Pos cover = 1 Neg cover = 0]
class_mlj(d06060518302200819).
[pos-neg] [1]

[Rule 19] [Pos cover = 5 Neg cover = 0]
class_mlj(A) :-
   pagina_1(A,B),
   frame(B,C),
   height_very_small(C),
   pos_left(C),
   pos_middle(C).
[pos-neg] [5]

[Rule 20] [Pos cover = 1 Neg cover = 0]
class_mlj(d06060510430514318).
[pos-neg] [1]

[Rule 21] [Pos cover = 1 Neg cover = 0]
class_mlj(d06060510442014369).
[pos-neg] [1]

[Rule 22] [Pos cover = 1 Neg cover = 0]
class_mlj(d06060518322601004).
[pos-neg] [1]

[Rule 23] [Pos cover = 1 Neg cover = 0]
class_mlj(d06060518303100860).
[pos-neg] [1]

[Rule 24] [Pos cover = 1 Neg cover = 0]
class_mlj(d06053013254013020).
[pos-neg] [1]

[Training set performance]
           Actual
        +          -  
     + 110         0         110 
Pred 
     -  0         208        208 

       110        208        318 

Accuracy = 1.0
[Training set summary] [[110,0,0,208]]

[Test set performance]
          Actual
       +        - 
     + 12        0        12 
Pred 
     - 0        23        23 

       12        23        35 

Accuracy = 1.0
[Test set summary] [[12,0,0,23]]

 --- Fold 1

[theory]

[Rule 1] [Pos cover = 44 Neg cover = 0]
class_mlj(A) :-
   pagina_1(A,B),
   frame(B,C),
   on_top(C,D),
   height_large(D),
   width_large(D),
   width_medium_small(C).
[pos-neg] [44]

[Rule 2] [Pos cover = 26 Neg cover = 0]
class_mlj(A) :-
   pagina_1(A,B),
   frame(B,C),
   allineato_al_centro_verticale(C,D),
   on_top(D,E),
   width_medium_small(D),
   pos_left(D).
[pos-neg] [26]

[Rule 3] [Pos cover = 7 Neg cover = 0]
class_mlj(A) :-
   pagina_1(A,B),
   frame(B,C),
   tipo_misto(C),
   height_very_small(C),
   pos_upper(C).
[pos-neg] [7]

[Rule 4] [Pos cover = 16 Neg cover = 0]
class_mlj(A) :-
   pagina_1(A,B),
   frame(B,C),
   on_top(C,D),
   width_large(D),
   width_very_small(C).
[pos-neg] [16]

[Rule 5] [Pos cover = 28 Neg cover = 0]
class_mlj(A) :-
   pagina_1(A,B),
   altezza_pagina(B,842.0),
   frame(B,C),
   on_top(C,D),
   pos_right(D).
[pos-neg] [28]

[Rule 6] [Pos cover = 1 Neg cover = 0]
class_mlj(d06060510442014369).
[pos-neg] [1]

[Rule 7] [Pos cover = 20 Neg cover = 0]
class_mlj(A) :-
   pagina_1(A,B),
   frame(B,C),
   on_top(C,D),
   height_medium_large(D),
   pos_left(C),
   larghezza_pagina(B,595.0).
[pos-neg] [20]

[Rule 8] [Pos cover = 22 Neg cover = 0]
class_mlj(A) :-
   pagina_1(A,B),
   frame(B,C),
   width_medium_small(C),
   pos_right(C),
   frame(B,D),
   height_small(D).
[pos-neg] [22]

[Rule 9] [Pos cover = 1 Neg cover = 0]
class_mlj(d06060518525802442).
[pos-neg] [1]

[Rule 10] [Pos cover = 1 Neg cover = 0]
class_mlj(d06053013254013020).
[pos-neg] [1]

[Rule 11] [Pos cover = 1 Neg cover = 0]
class_mlj(d06060510445614393).
[pos-neg] [1]

[Rule 12] [Pos cover = 4 Neg cover = 0]
class_mlj(A) :-
   pagina_1(A,B),
   frame(B,C),
   height_very_large(C),
   width_large(C).
[pos-neg] [4]

[Rule 13] [Pos cover = 1 Neg cover = 0]
class_mlj(d06060518322601004).
[pos-neg] [1]

[Rule 14] [Pos cover = 1 Neg cover = 0]
class_mlj(d06060518302200819).
[pos-neg] [1]

[Rule 15] [Pos cover = 1 Neg cover = 0]
class_mlj(d06060518303100860).
[pos-neg] [1]

[Rule 16] [Pos cover = 1 Neg cover = 0]
class_mlj(d06053013365813708).
[pos-neg] [1]

[Rule 17] [Pos cover = 1 Neg cover = 0]
class_mlj(d06060518304000869).
[pos-neg] [1]

[Rule 18] [Pos cover = 1 Neg cover = 0]
class_mlj(d06060518540302460).
[pos-neg] [1]

[Rule 19] [Pos cover = 1 Neg cover = 0]
class_mlj(d06060518272200636).
[pos-neg] [1]

[Rule 20] [Pos cover = 1 Neg cover = 0]
class_mlj(d06060510273413645).
[pos-neg] [1]

[Training set performance]
           Actual
        +          -  
     + 110         0         110 
Pred 
     -  0         208        208 

       110        208        318 

Accuracy = 1.0
[Training set summary] [[110,0,0,208]]

[Test set performance]
          Actual
       +        - 
     + 11        1        12 
Pred 
     - 1        22        23 

       12        23        35 

Accuracy = 0.942857142857143
[Test set summary] [[11,1,1,22]]
\end{verbatim}
\dots
\subsubsection{ALEPH}
\begin{verbatim}
--- Fold 0
class_elsevier(A) :-
   pagina_1(A,B), 
   frame(B,C), 
   on_top(C,D), 
   allineato_al_centro_verticale(D,E),
   tipo_linea_orizzontale(E), 
   tipo_misto(C).
class_elsevier(A) :-
   pagina_1(A,B), 
   frame(B,C), 
   altezza_rettangolo(C,0.07363421).
class_elsevier(A) :-
   pagina_1(A,B), 
   larghezza_pagina(B,544.0).

[Test set performance]
        Actual
       +        - 
     + 6        0        6 
Pred 
     - 0        29        29 

       6        29        35

Accuracy = 1.0
[Test set summary] [[6,0,0,29]]

 --- Fold 1
class_elsevier(A) :-
   pagina_1(A,B), 
   frame(B,C), 
   larghezza_rettangolo(C,0.0605042).
class_elsevier(A) :-
   pagina_1(A,B), 
   larghezza_pagina(B,544.0).
class_elsevier(A) :-
   pagina_1(A,B), 
   frame(B,C), 
   allineato_al_centro_orizzontale(C,D), 
   tipo_linea_orizzontale(D).
class_elsevier(A) :-
   pagina_1(A,B), 
   frame(B,C), 
   altezza_rettangolo(C,0.07363421).

[Test set performance]
          Actual
       +        - 
     + 6        0        6 
Pred 
     - 0        29        29 

       6        29        35 

Accuracy = 1.0
[Test set summary] [[6,0,0,29]]
\end{verbatim}
\dots
\subsection{Metriche}
\label{metriche}
Tra le informazioni presenti nei vari file di output è presente anche la matrice di confusione relativi a come sono stati classificati gli esempi di test.

La matrice di confusione sarà composta da:
\begin{table}[htbp]
\centering
\begin{tabular}{c c c c }
	& & \multicolumn{2}{ c }{\textit{Reali}} \\
	& & + & - \\
	\multirow{2}{*}{\textit{Predetti}} & + & TP & FP \\
	& - & FN & TN \\
\end{tabular}
\end{table}


Partendo da queste matrici di confusione ottenute, sono state calcolate delle metriche al fine di verificare quale algoritmo risulti essere più efficiente in termini di classificazione.
Le metriche calcolate sono state:
\begin{itemize}
	\item Precision
	\item Recall
	\item F-Measure
	\item Accuracy
	\item Error
\end{itemize}
Dove la Precision e la recall saranno calcolate rispettivamente dalle formule:
	$$Precision = \frac{TP}{TP + FP}$$
	$$Recall = \frac{TP}{TP + FN}$$
Per quanto riguarda la F-measure invece, è stata calcolata mettendo in relazione entrambe le metriche precedenti tramite la relazione:
$$F = 2*\frac{Precision * Recall}{Precision + Recall}$$
L'accuratezza invece è stata calcolata tramite la formula:
$$Acc = \frac{TP+TN}{TP + TN + FP + FN}$$
Infine è stata calcolato anche l'errore in base all'accuratezza della classificazione:
$$Err = 1 - Acc$$
