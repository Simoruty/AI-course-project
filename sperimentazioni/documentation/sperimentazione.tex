\section{Sperimentazione}

\subsection{I file eseguibili}
Per ogni fold di ogni dataset, abbiamo creato in maniera automatica uno script \verb+.yap+ che, se eseguito, si occupa di avviare ALEPH, leggere i file di input, generare le regole e scriverle su file.

Il sorgente dello script (esempio tratto dal fold 0 del dataset elsevier) è il seguente:

\begin{prologcode}
#!/usr/local/bin/yap -L --
#
# .
:- consult('../aleph.pl').
:- read_all('elsevier_f0').
:- induce.
:- write_rules('elsevier_f0.rul').
\end{prologcode}

Tutti i $40$ ($4$ dataset $\times$ $10$ fold) script \verb+.yap+ sono avviati in successione da un unico file scritto in linguaggio \emph{Python}.

Esso si occupa anche di tenere traccia dei tempi di esecuzione per un confronto sulle performance dei sistemi confrontati e di scrivere l'output delle esecuzioni in file \verb+.out+.

Il codice dello script è qui riportato:

\begin{pythoncode}
#!/usr/bin/python

import sys
import os
import subprocess
from time import gmtime, strftime, localtime
from datetime import datetime

cmd = "chmod +x **/*.yap"
subprocess.call(cmd, shell=True)
if not os.path.exists("./result/"):
    os.makedirs("./result/")
for dataset in ["elsevier", "jmlr", "mlj", "svln"]:
    print dataset +" started at "+ strftime("%H:%M:%S", localtime())
    sys.stdout.flush()
    startDataset = datetime.now()
    for fold in range(10):
        startTime = datetime.now()
        print "Fold " + str(fold) +" started at "
        print strftime("%H:%M:%S", localtime())
        sys.stdout.flush()
        cmd = "./"+dataset+"/"+dataset+"_f"+str(fold)
        cmd += ".yap -s50000 -h200000 2>&1 > /dev/null"
        subprocess.call(cmd, shell=True)
        fin = open("./result/"+dataset+".summary","a")
        fin.write("\nFold "+str(fold)+"\n")
        fin.close()
        cmd0 = "cat ./"+dataset+"/"+dataset+"_f"+str(fold)
        cmd0 += ".rul >> ./result/"+dataset+".summary"
        subprocess.call(cmd0, shell=True)
        cmd1 = "cat ./"+dataset+"/"+dataset+"_f"+str(fold)
        cmd1 += ".log | grep -B 10 -m 1 '\[Test set summary\]'"
        cmd1 += " >> ./result/"+dataset+".summary"
        subprocess.call(cmd1, shell=True)
        print "Fold " + str(fold) +" ended in: "
        print str(datetime.now()-startTime)
        sys.stdout.flush()
    print dataset+" ended in "+str(datetime.now()-startDataset)
    sys.stdout.flush()
\end{pythoncode}


\subsection{Hardware utilizzato}

\subsection{Metriche}
\label{metriche}
Partendo dalle matrici di confusione ottenute dall'esecuzione delle varie sperimentazioni, sono state calcolate delle metriche al fine di verificare quale algoritmo risulti essere più efficiente in termini di classificazione.
Le metriche calcolate 