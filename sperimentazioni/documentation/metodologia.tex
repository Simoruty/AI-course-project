\section{Metodologia Sperimentale}


\subsection{10-Fold Validation}

Per l'esecuzione dell'esperimento abbiamo scelto di utilizzare una $k$-fold validation con $k=10$, come comunemente usato in letteratura.
Per la scelta dei $10$ fold di ogni dataset si è proceduto con un campionamento stratificato. In ogni fold abbiamo mantenuto la stessa proporzione tra esempi positivi ed esempi negativi presente nel dataset completo, secondo lo schema riportato nelle tabelle  \ref{tab:Elsevier}, \ref{tab:JMLR}, \ref{tab:MLJ} e \ref{tab:SVLN}.

\begin{table}[h!tbp]
	\centering
		\begin{tabular}{l@{\qquad}*{10}{r}}
		\toprule
\addlinespace
			Fold &  0 &  1 &  2 &  3 &  4 &  5 &  6 &  7 &  8 &  9 \\
\addlinespace
\midrule
\addlinespace
\#Positive  & 6  & 6  &  6 &  6 &  6 &  6 &  6 &  6 &  6 &  7 \\
\#Negative  & 29 & 29 & 29 & 29 & 29 & 29 & 29 & 30 & 30 & 29 \\
\#Examples & 35 & 35 & 35 & 35 & 35 & 35 & 35 & 36 & 36 & 36 \\
\addlinespace
\bottomrule
		\end{tabular}
		\caption[Elsevier: suddivisione in fold.]{Suddivisione in fold del dataset Elsevier, mantendo le proporzioni tra esempi positivi e negativi.}
	\label{tab:Elsevier}
\end{table}

\begin{table}[h!tbp]
\centering
\begin{tabular}{l@{\qquad}*{10}{r}}
		\toprule
\addlinespace
Fold &  0 &  1 &  2 &  3 &  4 &  5 &  6 &  7 &  8 &  9 \\
\addlinespace
\midrule
\addlinespace
\#Positive  & 10  & 10  &  10 &  10 &  10 &  10 &  10 &  10 &  10 &  10 \\
\#Negative  & 25 & 25 & 25 & 25 & 25 & 25 & 25 & 26 & 26 & 26 \\
\#Examples  & 35 & 35 & 35 & 35 & 35 & 35 & 35 & 36 & 36 & 36 \\
\addlinespace
\bottomrule
\end{tabular}
		\caption[JMLR: suddivisione in fold.]{Suddivisione in fold del dataset JMLR, mantendo le proporzioni tra esempi positivi e negativi.}
	\label{tab:JMLR}
\end{table}


\begin{table}[h!tbp]
	\centering
		\begin{tabular}{l@{\qquad}*{10}{r}}
		\toprule
\addlinespace
			Fold &  0 &  1 &  2 &  3 &  4 &  5 &  6 &  7 &  8 &  9 \\
\addlinespace
\midrule
\addlinespace
\#Positive  & 12  & 12  &  12 &  12 &  12 &  12 &  12 &  12 & 13 & 13 \\
\#Negative  & 23 & 23 & 23 & 23 & 23 & 23 & 23 & 24 & 23 & 23 \\
\#Examples  & 35 & 35 & 35 & 35 & 35 & 35 & 35 & 36 & 36 & 36 \\
\addlinespace
\bottomrule
		\end{tabular}
\caption[MLJ: suddivisione in fold.]{Suddivisione in fold del dataset MLJ, mantendo le proporzioni tra esempi positivi e negativi.}
	\label{tab:MLJ}
\end{table}

\begin{table}[h!tbp]
	\centering
		\begin{tabular}{l@{\qquad}*{10}{r}}
		\toprule
\addlinespace
			Fold &  0 &  1 &  2 &  3 &  4 &  5 &  6 &  7 &  8 &  9 \\
\addlinespace
\midrule
\addlinespace
\#Positive & 7  & 7  &  7 &  7 &  7 &  7 &  7 &  7 &  7 &  7 \\
\#Negative & 28 & 28 & 28 & 28 & 28 & 28 & 28 & 29 & 29 & 29 \\
\#Examples & 35 & 35 & 35 & 35 & 35 & 35 & 35 & 36 & 36 & 36 \\
\addlinespace
\bottomrule
		\end{tabular}
\caption[SVLN: suddivisione in fold.]{Suddivisione in fold del dataset SVLN, mantendo le proporzioni tra esempi positivi e negativi.}
	\label{tab:SVLN}
\end{table}

Per assicurare la replicabilità dell'esperimento in appendice (\ref{appendix:fold}) è riportata l'esatta suddivisione in fold dei dataset.