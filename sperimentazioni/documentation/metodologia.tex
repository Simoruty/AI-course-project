\section{Metodologia Sperimentale}


\subsection{10-Fold Validation}

Per l'esecuzione dell'esperimento abbiamo scelto di utilizzare una $k$-fold validation con $k=10$, come comunemente usato in letteratura.
Per la scelta dei $10$ fold di ogni dataset si è proceduto con un campionamento stratificato. In ogni fold abbiamo mantenuto la stessa proporzione tra esempi positivi ed esempi negativi presente nel dataset completo.
Per assicurare la replicabilità dell'esperimento in appendice



\subsubsection{Elsevier}

\begin{table}[h!tbp]
	\centering
		\begin{tabular}{l@{\qquad}*{10}{r}}
		\toprule
\addlinespace
			Fold &  0 &  1 &  2 &  3 &  4 &  5 &  6 &  7 &  8 &  9 \\
\addlinespace
\midrule
\addlinespace
\#Positive  & 6  & 6  &  6 &  6 &  6 &  6 &  6 &  6 &  6 &  7 \\
\#Negative  & 29 & 29 & 29 & 29 & 29 & 29 & 29 & 30 & 30 & 29 \\
\#Examples & 35 & 35 & 35 & 35 & 35 & 35 & 35 & 36 & 36 & 36 \\
\addlinespace
\bottomrule
		\end{tabular}
	\label{tab:Elsevier}
\end{table}


\subsubsection{JMLR}

\begin{table}[h!tbp]
\centering
\begin{tabular}{l@{\qquad}*{10}{r}}
		\toprule
\addlinespace
Fold &  0 &  1 &  2 &  3 &  4 &  5 &  6 &  7 &  8 &  9 \\
\addlinespace
\midrule
\addlinespace
\#Positive  & 10  & 10  &  10 &  10 &  10 &  10 &  10 &  10 &  10 &  10 \\
\#Negative  & 25 & 25 & 25 & 25 & 25 & 25 & 25 & 26 & 26 & 26 \\
\#Examples  & 35 & 35 & 35 & 35 & 35 & 35 & 35 & 36 & 36 & 36 \\
\addlinespace
\bottomrule
\end{tabular}
	\label{tab:JMLR}
\end{table}



\subsubsection{MLJ}

\begin{table}[h!tbp]
	\centering
		\begin{tabular}{l@{\qquad}*{10}{r}}
		\toprule
\addlinespace
			Fold &  0 &  1 &  2 &  3 &  4 &  5 &  6 &  7 &  8 &  9 \\
\addlinespace
\midrule
\addlinespace
\#Positive  & 12  & 12  &  12 &  12 &  12 &  12 &  12 &  12 & 13 & 13 \\
\#Negative  & 23 & 23 & 23 & 23 & 23 & 23 & 23 & 24 & 23 & 23 \\
\#Examples  & 35 & 35 & 35 & 35 & 35 & 35 & 35 & 36 & 36 & 36 \\
\addlinespace
\bottomrule
		\end{tabular}
	\label{tab:MLJ}
\end{table}



\subsubsection{SVLN}

\begin{table}[h!tbp]
	\centering
		\begin{tabular}{l@{\qquad}*{10}{r}}
		\toprule
\addlinespace
			Fold &  0 &  1 &  2 &  3 &  4 &  5 &  6 &  7 &  8 &  9 \\
\addlinespace
\midrule
\addlinespace
\#Positive & 7  & 7  &  7 &  7 &  7 &  7 &  7 &  7 &  7 &  7 \\
\#Negative & 28 & 28 & 28 & 28 & 28 & 28 & 28 & 29 & 29 & 29 \\
\#Examples & 35 & 35 & 35 & 35 & 35 & 35 & 35 & 36 & 36 & 36 \\
\addlinespace
\bottomrule
		\end{tabular}
	\label{tab:SVLN}
\end{table}
